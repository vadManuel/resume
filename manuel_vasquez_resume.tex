\documentclass{article}
\usepackage[left=1cm, right=1cm, vmargin=1cm]{geometry}
\usepackage[shortlabels,inline]{enumitem}
\usepackage{amsmath, amssymb, array, booktabs, multicol}

\newcommand{\name}{Manuel Vasquez}
\newcommand{\website}{http://www.vadmanuel.com}
\newcommand{\email}{vadmanuel@knights.ucf.edu}
\newcommand{\phone}{(407) 914-0175}
\newcommand{\makeminipage}[4]{
    \begin{minipage}[c]{9cm}
        \flushleft #1 \\ #2
    \end{minipage} \hfill
    \begin{minipage}[c]{9cm}
        \flushright #3 \\ #4
    \end{minipage}
}
\newcommand{\makesection}[1]{\hrule\vskip1mm\uppercase{#1}\vskip1mm\hrule}
\pagenumbering{gobble}

\begin{document}
    \setlength{\parindent}{0cm}

    \makeminipage
        {{\large\textbf{\name}}}
        {\website}
        {Email: \email}
        {Phone: \phone}
    \bigbreak

    \makesection{Education}
    \begin{itemize}[leftmargin=.35cm]
        \item \makeminipage
            {\textbf{University of Central Florida}}
            {\textit{Bachelor of Engineering in Computer Science; GPA: 3.00}}
            {\textit{Orlando, FL}}
            {\textit{Summer 2020}}
    \end{itemize}

    \makesection{Projects}
    \begin{itemize}[leftmargin=.35cm]
        \item \textbf{UCF101-Sports dataset action recognition} \\
        Utilized Google's Inception network to extract features and passed them to an RNN
        in order to maintain spatiotemporal information. Due to the size of the dataset, I
        used LOO cross-validation and achieved a validation accuracy of 72.87\% on the 10
        different actions.
        \item \textbf{DJI Tello active hand tracking} \\
        Gave a Tello drone the ability to be controlled by the palm of a single user. Utilizing
        the 11k Hands database by Mahmoud, I was able to train a model to track the palm of the
        user. The drone maintains the palm of the user centered in its field of view.
        \item \textbf{\textit{KnightyKnights.com} a MERN stack contact manager} \\
        Our team was comprised of 5 members and had 20 days to build a contact manager with the
        web stack of our choice, we agreed to implement a MERN stack. As the team project manager,
        I implemented the Agile methodology in order to accept changes that might come later on.
        In order to speed up the building phase of each step we all learned each component of
        the stack and branched our submissions on Github.
        \item \textbf{MNIST dataset handwritten digit recognition} \\
        Using a ConvNet achieved 99.23\% validation accuracy using a 30\% validation set.
        Trained with SGD (learning rate=0.1) with a total of 60 epochs, mini-batch size of 10,
        and Dropout (rate=0.5) for regularization.
        \item \textbf{Kinetics-600 dataset action recognition using I3D} \\
        A recent project in which I move forward from older technologies. I attempt to
        decompose and re-engineer the Two-Stream Inflated 3D ConvNet created by Carreira
        and Zisserman.
        \item \textbf{Dynamically aimed refreshments cooler with skeleton tracking} \\
        A current project where I use optical flow to lower the computational cost of tracking
        more than a single skeleton with a Kinect V2. The project's objective is to track the
        user, count the number of fingers they are holding up and launch a can from the
        appropriate compartment to the user at a variable distance.

    \end{itemize}
    \makesection{Clubs | Activities | Certification}
    \begin{itemize}[leftmargin=.35cm]
        \item \textbf{UCF Robotics Club} Laki2 Drone Payload Delivery Team \\
        This drone is the Robotics Club's first entry into the AUVSI SUAS competition. My task
        in this project was to create shape and character recognition models. We ran into data
        issues from the beginning since all we had was artificially created pictures of the
        landscape, so I was also tasked with creating an image generator as a method to train
        the relaxed shape and character recognition models.
        \item \textbf{AI@UCF} Fall and Spring semester Coordinator position candidate
        \vspace*{-.05cm}
        \begin{itemize}[$\circ$]
            \item \textbf{Intelligence:} We were tasked with reading and analysing a research
            paper weekly. Once a week we met to discuss issues, questions, or simply what we
            thought about the paper.
            \item \textbf{Course:} The club met once a week to discuss and implement practical
            applications of the current week's topic. We would then code up working examples on
            Google Colab.
        \end{itemize}
        \item \textbf{UCF Programming Team:} Members meet every Saturday morning for a 2-hour lecture
        and 4-hour mock contest.
        \item \textbf{Coursera Machine Learning by Andrew Ng}
        \item \textbf{SHPE} Member, \textbf{Knight Hacks} 48-hour participant, and granted access to the \textbf{Stokes cluster}
    \end{itemize}
        
    \makesection{Programming Languages and Technologies}
    \vspace*{-2mm}
    \begin{multicols}{2}\begin{itemize}[leftmargin=.35cm]
        \setlength\itemsep{0cm}
        \item Arduino C, C++, C\#, \LaTeX, Java, Javascript, Python
        \item Amazon Web Services
        \item Google Cloud Services
        \item Keras TensorFlow
        \item PyTorch
        \item \textbf{M}ongoDB, \textbf{E}xpress.js, \textbf{R}eact.js, \textbf{N}ode.js
        \item Redux.js
        \item React Native framework
    \end{itemize}\end{multicols}
\end{document}